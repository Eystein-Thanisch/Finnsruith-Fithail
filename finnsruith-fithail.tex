\documentclass[11pt]{article}

\usepackage{xltxtra}
\setmainfont[Mapping=tex-text]{Linux Libertine O}
\usepackage{polyglossia}
\setmainlanguage[variant=british]{english}

\usepackage{ledmac}
\usepackage{ledpar}
\usepackage{csquotes}

\author{Christopher Guy Yocum}
\title{\emph{Finn\.{s}ruith F\'{i}thail}: Text A (TCD H. 3. 18)}

\footparagraph{A}

\begin{document}

\maketitle

The \emph{Finn\.{s}ruith F\'{i}thail} is the first and oldest text to contain any traces of the early Irish judge F\'{i}thal.  Moreover, it is an early Irish legal text which has remained without a translation or a critical edition.  This draft translation moves forward knowledge both of F\'{i}thal in particular but early Irish law more generally and allows scholars to delve confidently into the particulars of early Irish law.

\emph{Finn\.{s}ruith F\'{i}thail} has a diplomatic transcription by D.A. Binchy in his magisterial \emph{Corpus Iuris Hibernici} where it appears in two versions: A (CIH786.25--789.17) and B (CIH 2131.1--2143.40).  The translation offered here is based on Binchy's transcription of the A text.

\Large \textbf{THIS IS A FIRST DRAFT TRANSLATION} as such it should be treated with extreme caution.  There will be errors and omissions of all forms.  Some of these errors may be quite amusing.  Be that as it may, the translation will be updated from time to time with more and better translations and notes.

\newpage

\begin{pages}
  \begin{Leftside}
    \beginnumbering\pstart

    A mo \edtext{sruith}{\Afootnote{i-stem masc. \enquote{elder, venerable}}} .i. ol socht. \edtext{Co}{\Afootnote{gem. conj. part. \enquote{how} GOI p. 290}} \edtext{ber}{\Afootnote{1st pers. pres. subj. conj. or 1st person fut. indict. act. (no length mark)}} br\emph{eith} .i. \edtext{cionnus}{\Afootnote{from cindas (d) \enquote{how?}}} \edtext{berat}{\Afootnote{3rd per. pres./fut. subj. conj. \enquote{should/shall bear}}} \edtext{b\emph{r}eatha}{\Afootnote{\={a}-stem f. acc. pl. \enquote{judgements}}}.  \edtext{b\emph{er}o}{\Afootnote{1st pers. subj./fut. act. \enquote{I shall/should bear}}} \edtext{nip}{\Afootnote{3rd pers. pres. subj. neg. of is \enquote{so that ti may not be}}} \edtext{sl\emph{an}}{\Afootnote{DIL S 260 \enquote{exempt, non-liable, safe}}}.i. ar fithal.

    \pend
    \endnumbering
  \end{Leftside}

  \begin{Rightside}
    \beginnumbering\pstart
    
    \enquote{Oh my elder} .i. says Socht.  How should/shall I judge .i. how does/shall he bear judgements.  I should/shall bear [judgement] such that it may not be exempt .i. says F\'{i}thal.
  
    \pend
    \endnumbering
  \end{Rightside}

  \Pages


\end{pages}

\end{document}
